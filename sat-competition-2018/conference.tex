%\documentclass[runningheads]{llncs}
\documentclass[final]{ieee}

\usepackage{microtype} %This gives MUCH better PDF results!
%\usepackage[active]{srcltx} %DVI search
\usepackage[cmex10]{amsmath}
\usepackage{amssymb}
\usepackage{fnbreak} %warn for split footnotes
\usepackage{url}
%\usepackage{qtree} %for drawing trees
%\usepackage{fancybox} % if we need rounded corners
%\usepackage{pict2e} % large circles can be drawn
%\usepackage{courier} %for using courier in texttt{}
%\usepackage{nth} %allows to \nth{4} to make 1st 2nd, etc.
%\usepackage{subfigure} %allows to have side-by-side figures
%\usepackage{booktabs} %nice tables
%\usepackage{multirow} %allow multiple cells with rows in tabular
\usepackage[utf8]{inputenc} % allows to write Faugere correctly
\usepackage[bookmarks=true, citecolor=black, linkcolor=black, colorlinks=true]{hyperref}
\hypersetup{
pdfauthor = {Alexander Scheel},
pdftitle = {CryptoMiniSat v5.5},
pdfsubject = {SAT Competition 2018},
pdfkeywords = {SAT Benchmarks, Cryptography, Logical Cryptanalysis},
%pdfcreator = {PdfLaTeX with hyperref package},
%pdfproducer = {PdfLaTex}
}
%\usepackage{butterma}

%\usepackage{pstricks}
\usepackage{graphicx,epsfig,xcolor}

\begin{document}
\title{Logical Cryptanalysis Benchmarks for \\ Classical and Modern Hash Functions}
\author{Alexander Scheel, Iowa State University, alexander.m.scheel@gmail.com}

\maketitle
\thispagestyle{empty}
\pagestyle{empty}

\section{Introduction}
This collection of benchmarks focuses on new techniques in logical
cryptanalysis for analyzing the structure of various hash functions. We
offer as a benchmark a new technique analyzing the security of a hash function
of the Merkle-D{\aa}mgard construction which are broadly applicable to all such
constructions and several benchmarks analyzing the various security properties
of the Keccak hash function. While the latter techniques have not yeilded
useful cryptanalysis results, they pose as a source of variable scalable
benchmarks due to several choices of parameters (the bitwidth, $w$ and the
number of rounds).

\section{Classical Hash Functions}
The MD4 hash function is composed of 48 iterated rounds, each updating one of
four 32-bit state variables \cite{rfc1320}. While authors such as I. Mironov
have applied SAT solvers to finding collisions given an existing differential
path \cite{Mironov2006} and D. Jovanovi{\'{c}} has applied applied SAT solvers
to brute force finding collisions and preimages in hash functions
\cite{LogicalAnalysis}, these techniques either require pre-existing
collisions or are too hard for SAT solvers on a large number of rounds.

One technique that is efficient for even a modest number of rounds ($\leq 28$)
is the notion of a differential family search. In most cases, the differential
path between two blocks $b_1$ and $b_2$ is the simple XOR difference between
the intermediate rounds. For a given path $p$, there are often many such blocks
which create a collision; I. Mironov showed that SAT solvers can find such
blocks in MD4 and MD5 \cite{Mironov2006} relatively quickly. We can extend this
concept to differential paths as well: given a differential path $p$, its
family is the ordered tuple of indices of the rounds with a non-zero
difference. For example, the family of the differential path introduced by
M. Schlaffer (in \cite{Schlaffer2006}) is:
\begin{align*}
    (1, 2, 3, 5, 6, 7, 8, 9, 10, 11, 12, 15, 16, 19, 20, 35, 36)
\end{align*}
Thus for a given family, $f$, there are multiple possible differential paths
which have a structure described by $f$.

By using this concept of a differential family, we can divide the search space
into multiple different SAT problems: for a given number of rounds, $r$, there
are $2^{r-4}$ possible differential families (versus $2^{32w-32}$ possible
differential paths). For $r \leq 28$ it is possible to exhaustively search
large portions of the family space; for $32 \leq r \leq 36$ it becomes possible
to find in select cases, and for $r > 36$, it remains impossible in nearly all
cases (due to the large time limit). Our benchmarks are a sample of possible
differential families for $r=24$, mixing both SAT and UNSAT results. These
benchmarks can be found in the \texttt{families} folder \cite{Sat2018}.


\section{Modern Hash Functions}

The notion of a differential family is not as useful for Keccak due to
the interaction between the sponge function and the internal permutation
functions causing the collision space of families to be entirely different
between successive rounds. From an algebraic perspective, and to study the
security margin of the XOF consruct \cite{Keccak3}, it is important to show
that the Keccak round functions are bijective and have high permutation order.
Two of our benchmarks (\texttt{bijection} and \texttt{orders}) model these
problems in SAT. In general, these instances are easy, but with a few notable
outliers: for $w \geq 4$, proving the order of $\theta$ is $3w$ is difficult.
Note that this should simplify to showing that $\forall x, x \neq x$ is UNSAT,
and thus should be relatively trivial; however $w=4$ produces runtimes in
excess of one hour. Further, anything later involving $\theta$ (such as
$\rho \circ \theta$, etc.) also becomes difficult for $w\geq4$.

In \texttt{differences}, we study the differential properties of the
permutation functions: for a given parameters $x, y \leq 25w$ and round
function $f$, we seek to find witnesses $a$ and $b$ such that:
\begin{align*}
    & \#(1, a \oplus b) = x \\
    & \#(1, f(a) \oplus f(b)) = y
\end{align*}
That is, find an input with difference $x$ which produces an output with
difference $y$. There are a few interesting outliers in this model: for
$w\geq4$, most round functions cannot be individually analyzed this way.
Further, while $\pi$ is a permutation of the order of the bits (and thus
does not change the values of any bits), certain instances in $w=2$
where $x \neq y$ produce runtimes in excess of an hour. This is surprising
as the model is trivially SAT if and only if $x == y$ (and the model merely
involves changing the orders of variables).

The benchmarks in \texttt{output-margins} consider the effects of the sponge
function. Since all of the round functions are bijective, if two inputs differ,
the interal state of Keccak must also differ. However, to produce a collision,
only the first $y$ bits of the output must be the same. Thus we can create
a model for a given number of differences $x, z \leq 25w$, security margin
$y \leq 25w$, and round functions $f$:
\begin{align*}
    & \#(1, a xor b) = x \\
    & \#(1, (f(a) xor f(b))[0:y]) = z
\end{align*}
If such a witness a and b exist for $z=0$, then the input difference $x$ is
possible of producing collisions at a security margin $m$. Our provided
benchmarks sample the space for small values of $w$ and relatively few
round functions; for $w\geq 16$ and for any set of functions including
$\theta$, the runtimes become exceedingly long.

The benchmarks in \texttt{xof-state} attempt to recreate the internal state
of Keccak given a series of outputs from the XOF (extensible output function
at a given margin). In general, this is possible for either small values of
$w$ or small numbers of rounds (for larger values of $w$). However, care must
be selected in choosing the base seed, otherwise, there can possibly be
multiple satisfying seeds. However, for a set of inputs with unique solution,
these benchmarks can be extended to contain the output from several rounds of
Keccak. After a threshhold dependent on the margin, these are redundant
information and thus test the SAT solvers to work with larger models which
overfit to the solution.

\section{Thanks}
A special thanks to Mate Soos and his CryptoMiniSat solver
\cite{CryptoMiniSat5} for suggesting these problems be submitted as benchmarks
to the conference.

This work was completed as part of an Undergraduate Honors Research project
at Iowa State University under the guidance of Eric W. Davis and as part of
MATH 490: Independent Study of SHA-3 under the guidance of Clifford Bergman.

\bibliographystyle{splncs03}
\bibliography{competition.bib}

\vfill
\pagebreak

\end{document}
