% TEMPLATE for Usenix papers, specifically to meet requirements of
%  USENIX '05
% originally a template for producing IEEE-format articles using LaTeX.
%   written by Matthew Ward, CS Department, Worcester Polytechnic Institute.
% adapted by David Beazley for his excellent SWIG paper in Proceedings,
%   Tcl 96
% turned into a smartass generic template by De Clarke, with thanks to
%   both the above pioneers
% use at your own risk.  Complaints to /dev/null.
% make it two column with no page numbering, default is 10 point

% Munged by Fred Douglis <douglis@research.att.com> 10/97 to separate
% the .sty file from the LaTeX source template, so that people can
% more easily include the .sty file into an existing document.  Also
% changed to more closely follow the style guidelines as represented
% by the Word sample file.

% Note that since 2010, USENIX does not require endnotes. If you want
% foot of page notes, don't include the endnotes package in the
% usepackage command, below.

\documentclass[letterpaper,twocolumn,10pt]{article}
\usepackage{usenix,epsfig,endnotes,listings,amsmath,xcolor,courier,float}
\usepackage{hyperref,supertabular}

\lstset{
  basicstyle=\ttfamily,
  columns=fullflexible,
  frame=single,
  breaklines=true,
  postbreak=\mbox{\textcolor{red}{$\hookrightarrow$}\space},
}

\begin{document}

%don't want date printed
\date{}

%make title bold and 14 pt font (Latex default is non-bold, 16 pt)
\title{\Large \bf Motivating a Constraint-Based Approach to Colliding Hash Functions }

\author{
{\rm Alexander Scheel}\\
\href{mailto:alexander.m.scheel@gmail.com}{alexander.m.scheel@gmail.com} \\
Iowa State University
\and
{\rm Eric Rozier}\\
\href{mailto:erozier@iastate.edu}{erozier@iastate.edu} \\
Iowa State University
}

\maketitle

% Use the following at camera-ready time to suppress page numbers.
% Comment it out when you first submit the paper for review.
% \thispagestyle{empty}


\subsection*{Abstract}

TODO

\section{Introduction}
\subsection{Brief History of SAT in Cryptography}

Much of the original research for developing the use of SAT with cryptography
was done by Fabio Massacci currently at the University of Trento \cite{CryptoSAT}.
The focus of this work was largely two fold: cryptoanalysis against the DES
and faking signatures in the RSA with $e=3$. More information can be found
here \cite{MASSACCI2000LogicalCA} and here \cite{FIORINI2003101}, but note that
this work predates all of the collisions discussed in this article.


\begin{itemize}
    \item CryptoSAT
    \item cryptominisat
    \item SHA-1
\end{itemize}

\subsection{Motivating Logical Cryptanalysis on Hash Functions}

We wish to provide the motivations driving our work. To begin, while collisions
in cryptographic hash functions have a wide reach beyond the academic community,
most of the discovered collisions are driven by academics. As such, most papers
begin by introducing a single hash function, the author's preferred notation,
the methods used in whole or part used to discover the collision, the constraints
on the differential and the differential path, and usually end with an example of a
found collision. However, while community's goal is to improve techniques by
better differentials as measured by higher likelihoods of occurrence, little work
is done to determine the utility of the collision. We seek to push the
community in a new direction, towards measuring, and later improving, the utility
of collisions.

Thus we seek to answer the following questions: What does the input space of
a collision look like? Are blocks only of high entropy, likely reflecting
complex dependencies on bits of the input? Or, can fixed strings be required at
positions of an input? Can these blocks be constrained to only inputs of ASCII
or other character sets? Is the collision fixed to a single input differential,
or can many different input differentials be used? Are there constraints on
possible starting states?

By answering these questions, we begin to develop a sense of utility of
collisions. We can see whether they work well at the beginning or end of
longer messages; we can see how much flexibility we have to fit existing
data (towards second-preimage attacks matching existing collisions); and, we
can see how much structure we can enforce for members of a collision. However,
what we cannot answer are the relationships between these facets of utility,
and whether there are more facets we are overlooking.

Ultimately, we seek to develop a framework for pushing collisions from an
academic exercise to a practical exercise in a constrained environment. The
use is two fold: to develop better hash functions with highly-fractured collision
spaces, and to allow users of hash functions to understand the weaknesses
better. Our belief is that, while there should be an abundance of possible
collisions, hash functions can be designed so that possible collisions do not
pose security risks in all use cases. Further, we seek to demonstrate the
usefulness of discussing collisions in a common language of SAT. This will
allow properties of collisions to be easily verified, and find collision
examples targeting a specific use cases.

We leave it as an open challenge to find a set of collisions which lie outside
of our proposed methodology but which can be demonstrated to have high utility.

\subsection{Terminology of Utility}

We seek to extend the literature about the utility of collisions and thus
propose the following new terminology.

The \textbf{constraints} of a collision are the necessary and/or sufficient
boolean formulas for generating a collision, typically as a function of the
state blocks.

A \textbf{state transition constraint} is a constraint on the intermediate
state variables, i.e., round variables. A constraint is \textbf{weak} if
it is a any difference, and a constraint is \textbf{strong} if it is a
signed difference. A equality constraint can either be weak or strong. This
is roughly equivalent to a differential path.

\textbf{Input block constraints}, \textbf{input state constraints}, and
\textbf{output state constraints} are all analogously defined as state
transition constraints.

A \textbf{property} of a collision is any constraint placed upon the
input block which exists independent from the constraints of a collision.

An attack is \textbf{constraint-based} if there exists a SAT encoding of the
attack which can generate multiple collisions.

A collision is \textbf{extensible} if one example of a collision will lead to
other examples under the same differential and weak state transition constraints.
Note that if a collision is extensible, it is also constraint-based.

A collision is \textbf{differential-independent} if there are multiple
differentials which satisfy the state transition constraints.

A collision is \textbf{malleable} if collisions can be found with different
properties such as ASCII or JSON input blocks, or with fixed internal text.

A collision is \textbf{state-independent} if there exist alternative
starting states which produce collisions. A state-independent collision
is termed \textbf{strong} if one block can collide under two or more starting
states.

Thus, a collision is said to have \textbf{high utility} if it is
differential-independent, malleable, and state-independent.

\subsection{Common Notation for Constraint-Based Collisions}
A SAT model, $S = (E, A)$, is an ordered tuple consisting of a set, $E$, of
boolean equations of the form $var := expression$, and a set, $A$, of expressions
which must be asserted to true. It should be obvious that this is equivalent
to $3-CNF-SAT$. This notation is used as it is most similar to the ``circuits''
language \cite{circuits}; further, unline $3-CNF-SAT$, this notation is composable
in the sense that adding new constraints does not impact other existing constraints.

We now introduce a series of notations to ease construction of cryptographic
SAT models.

We define
\begin{align*}
    c_{l}^{l + k - j} := EQ_{i = j}^{k}(a_i, b_i)
\end{align*}
for $j \geq 0, k \geq j, l \geq 0$ to be equivalent to:
\begin{align*}
    \{ c_l & := a_j == b_j, \\
    c_{l+1} & := a_{j+1} == b_{j+1}, \\
    ..., \\
    c_{l+k-j} & := a_{k} == b_{k} \} \\
\end{align*}
In particular, $c$ is a tuple of boolean equations with cardinality $|c| = k-j$,
defining an equality relation between input sets $a$ and $b$, at index starting
at $j$ and ending at $k$ (inclusive).

Discuss encodings of hash functions? No, leave to section in

Use notation in other sections. Use Overview subsection to discuss
the particular quirks of encoding of hash function into SAT.

\section{MD4}
\subsection{Brief History and Overview}

The MD4 hashing algorithm was developed by Ron Rivest and first released as RFC 1186
in 1990 \cite{rfc1186}. This was later superseded by RFC 1320 in 1992 because the
reference implementation was ``more portable'' \cite{rfc1320}. As early as 2004,
Wang et al, demonstrated the first full collision in MD4 and MD5 \cite{cryptoeprint:2004:199}.
However, it wasn't until 2005 that details of the differential and constraints
were published \cite{Wang2005}. It is believed by some that these differentials
and constraints were found "by hand" \cite{cryptoeprint:2007:206}. Others thus
strove to automate these attacks with algorithmic approaches \cite{Schlaffer2006}
\cite{cryptoeprint:2007:206}, and \cite{Sasaki2007}.

In this section, we will analyze Wang's 2005 and Sasaki's 2007 differentials to MD4.

\subsection{SAT Model of MD4}
Base model.

\subsection{Wang's MD4 Collisions}
\subsubsection{Overview and Motivation}
While many have studied Wang's attack from the point of view of improving the
probability of collision or validating constraints, we seek to extend Wang's work
by showing properties about colliding blocks. In particular, we wish to show that
Wang's attack is constraint-based, extensible, highly malleable,
differential-independent, and state-independent. Thus, we venture that Wang's
attack is a strong attack against MD4, and that MD4 has no remaining collision
resistance.

\subsubsection{Extensible}
We note that Wang's attack is extensible. With a loose fitting on the original
example as referenced by `md4-wang-extensible', we generated an additional
40,000 new collisions. We note that, while not fully explored, the loose
fitting on the original hash produces valid collisions satisfying the closer
reconstruction of the original constraints; that is, no further differential
paths were discovered. By analyzing the new collisions for fixed properties,
an alternative construction of Wang's constraints can be created. Further,
by then negating individual constraints and testing for UNSAT, we can validate
that they are indeed necessary. See table \ref{table:1} for the set of differential
path constraints and table \ref{table:2} for the set of fixed-value constraints.
Together these form an equivalent constraint set as what is seen in \cite{Wang2005}.


\subsubsection{Constraint-Based}
\begin{itemize}
    \item Simple encoding
    \item Advanced constraints
    \item Time difference
\end{itemize}
testing
\subsubsection{Differential-Independent}
\subsubsection{State-Independent}
h1: f7f3fe7437e33bee3590392936aeab37e0b848329173bf0f08a252a0281e3284f4e4fd90626c4cbb3350ac3e36694d34703d31498806e520608c6738da70170b:40a8b995bd89b97662b4516e1df317f2
h2: f7f3fe7437e33b6e3590399936aeab37e0b848329173bf0f08a252a0281e3284f4e4fd90626c4cbb3350ac3e36694d34703d30498806e520608c6738da70170b:40a8b995bd89b97662b4516e1df317f2
h3: f7f3fe7437e33bee3590392936aeab37e0b848329173bf0f08a252a0281e3284f4e4fd90626c4cbb3350ac3e36694d34703d31498806e520608c6738da70170b:aabcc2e21ea0e529178ca7dcb21240a2
h4: f7f3fe7437e33b6e3590399936aeab37e0b848329173bf0f08a252a0281e3284f4e4fd90626c4cbb3350ac3e36694d34703d30498806e520608c6738da70170b:aabcc2e21ea0e529178ca7dcb21240a2
c Total time               : 127.12

\subsubsection{Malleable}
\begin{itemize}
    \item ASCII
    \item JSON
    \item substring
\end{itemize}
Testing
\subsubsection{Additional Notes}

\subsection{Sasaki's MD4 Collisions}
\subsubsection{Overview and Motivation}
We study Sasaki's collisions as a means of analyzing a more complete attack,
and as a point of comparison to Wang's \cite{Sasaki2007}. In particular, Sasaki boasts
of substantial speed improvement over Wang's and an increased probability of
collision. The latter should thus translate to improved results with respect to
malleability.

\subsubsection{Extensible}
We note that Sasaki's MD4 Collisions is an extensible collision. With a loose
fitting on the original example as referenced by `md4-sasaki-extensible',
we generated an additional 40,000 new collisions. Further, the constraints
published on page 16, while sufficient, are not necessary; many of the negated
forms have valid collisions as well \cite{Sasaki2007}.


\subsubsection{Differential-Independent}
We note that Sasaki's MD4 collisions are differential independent, finding
examples of all 64 theoretical differentials.

\subsubsection{State Independent}
h1: 13d340f35ff2c0ace19bcd73f5c153aa816105701b5050a2c7e0ab260adf14026e0a20b1f39a8f4be37e60aad37d5b1100140082c228676c4d1c8098c095d023:746094a6c1cf11f96614c893f6b54d67
h2: 13d340035ff2c0ace19bcdf3f5c153aa816105f01b5050a2c7e0ab260adf14026e0a2031f39a8f4be37e60aad37d5b1100140002c228676c4d1c8098c095d023:746094a6c1cf11f96614c893f6b54d67
h3: 13d340f35ff2c0ace19bcd73f5c153aa816105701b5050a2c7e0ab260adf14026e0a20b1f39a8f4be37e60aad37d5b1100140082c228676c4d1c8098c095d023:ef64866bf9eedb5a723d7153bcce781f
h4: 13d340035ff2c0ace19bcdf3f5c153aa816105f01b5050a2c7e0ab260adf14026e0a2031f39a8f4be37e60aad37d5b1100140002c228676c4d1c8098c095d023:ef64866bf9eedb5a723d7153bcce781f
c Total time               : 165.32



\section{MD5}
\subsection{Brief History and Overview}
\subsection{Wang's MD5 Collisions}
\subsection{Stevens's MD5 Collisions}
\subsection{Xie's MD5 Collisions}
\subsection{Sasaki's MD5 Collisions}

\section{SHA1}
\subsection{Brief History and Overview}
\subsection{Wang's SHA1 Partial Collision}
\subsection{Stevens's SHA1 Full Collision}

\newpage
\section{Collisions}
\subsection{Notation}
We propose the convention of using both a hexadecimal and binary representation.
Hex in a "standard order", i.e.,
\begin{lstlisting}
echo ``<hex>'' | xxd -r -p | openssl md4
\end{lstlisting}

\subsection{MD4}
\subsubsection{Wang's Collisions}

\newpage
\section{Tables}
\subsection{Notation}
Tabular notation notes go here.
Meaning of +, -, T, F, and *.
Note that missing numbers are replaced
with ... when all the same.

\subsection{MD4}
\subsubsection{Wang's Collision}

\begin{table}[H]
\centering
\begin{tiny}
\texttt{
\begin{tabular}{|c c|}
\hline
h1i & constraints \\
\hline
0 & ................................ \\
1 & .........................+...... \\
2 & .....................+..-....... \\
3 & ......+......................... \\
4 & ................................ \\
5 & ..................+............. \\
6 & ..........+-++.................. \\
7 & .................+--............ \\
8 & ...............+................ \\
9 & ......-..+--+................... \\
10 & ..-............................. \\
11 & +............................... \\
12 & ......+..+...................... \\
13 & ..+-.-.......................... \\
14 & ................................ \\
15 & .............+.................. \\
16 & -..-.+-......................... \\
17 & ................................ \\
18 & ................................ \\
19 & +.-............................. \\
20 & -.+-............................ \\
21 & ................................ \\
... & ................................ \\
47 & ................................ \\
\hline
\end{tabular}
}
\end{tiny}
\caption{Discovered differential path constraints in MD4 under Wang's Attack}
\label{table:1}
\end{table}


\begin{table}[H]
\centering
\begin{tiny}
\texttt{
\begin{tabular}{|c c|}
\hline
h1i & constraints \\
\hline
0 & .........................F...... \\
1 & .........................F...... \\
2 & .....................F..TT...... \\
3 & ......F..............F..FT...... \\
4 & ......F..............T..T....... \\
5 & ......T...........F............. \\
6 & ..........FTFF....F............. \\
7 & ..........FFFF...FTT............ \\
8 & ..........TFFF.F.TTT............ \\
9 & ......T..FTTF..F.TTT............ \\
10 & ..T...F..FFFF..T................ \\
11 & F.F...T..FTTF................... \\
12 & F.T...F..F...................... \\
13 & T.FT.TF..F...................... \\
14 & ..FF.FT..T...................... \\
15 & ..FT.TT......F.................. \\
16 & T..T.FT......................... \\
17 & ...T.TT......................... \\
18 & ...T.TT......................... \\
19 & F.TT............................ \\
20 & T.FT............................ \\
21 & ...T............................ \\
22 & ...T............................ \\
23 & ................................ \\
... & ................................ \\
34 & ................................ \\
35 & T............................... \\
36 & T............................... \\
37 & ................................ \\
... & ................................ \\
47 & ................................ \\
\hline
\end{tabular}
}
\end{tiny}
\caption{Discovered fixed-value constraints in MD4 under Wang's Attack}
\label{table:2}
\end{table}

\subsubsection{Sasaki's Collision}

\begin{table}[H]
\centering
\begin{tiny}
\texttt{
\begin{tabular}{|c c|}
\hline
h1i & constraints \\
\hline
0 & -.............................+- \\
0 & +.............................-+ \\
1 & ......................-.+....... \\
1 & ......................+.-....... \\
2 & ....................-........... \\
2 & ....................+........... \\
3 & ..........-++................... \\
3 & ..........+--................... \\
4 & .......-+-............-++++++... \\
4 & .......-++............-++++++... \\
4 & .......+--............+------... \\
4 & .......+-+............+------... \\
5 & -..........-++.................. \\
5 & -..........+--.................. \\
5 & +..........-++.................. \\
5 & +..........+--.................. \\
6 & -......-++............-++++++++. \\
6 & -......-++............+--------. \\
6 & -......+--............-++++++++. \\
6 & -......+--............+--------. \\
6 & +......-++............-++++++++. \\
6 & +......-++............+--------. \\
6 & +......+--............-++++++++. \\
6 & +......+--............+--------. \\
7 & ................................ \\
8 & ......-.-+...................... \\
8 & ......-.+-...................... \\
8 & ......+.-+...................... \\
8 & ......+.+-...................... \\
9 & ---.+--............-............ \\
9 & -++.-++............+............ \\
9 & -++.+--............+............ \\
9 & +--.-++............-............ \\
9 & +--.+--............-............ \\
9 & +++.-++............+............ \\
10 & ...................-............ \\
10 & ...................+............ \\
11 & ...................-............ \\
11 & ...................+............ \\
12 & ..-+..-......................... \\
12 & ..-+..+......................... \\
12 & ..+-..-......................... \\
12 & ..+-..+......................... \\
13 & ................................ \\
14 & ................................ \\
15 & -............................... \\
15 & +............................... \\
16 & ...-............................ \\
16 & ...+............................ \\
17 & ................................ \\
18 & ................................ \\
19 & ................................ \\
20 & -............................... \\
20 & +............................... \\
21 & ................................ \\
... & ................................ \\
31 & ................................ \\
32 & -............................... \\
32 & +............................... \\
33 & ................................ \\
... & ................................ \\
47 & ................................ \\
\hline
\end{tabular}
}
\end{tiny}
\caption{Discovered differential paths' constraints in MD4 under Sasaki's Attack.
Includes duplicate paths.}
\label{table:3}
\end{table}


\begin{table}[H]
\centering
\begin{tiny}
\texttt{
\begin{tabular}{|c c|}
\hline
h1i & constraints \\
\hline
0 & ......................T.T....... \\
1 & T..............................T \\
2 & T.....................F.F......F \\
3 & ....................F.T.T....... \\
4 & ..........FFF.......F........... \\
5 & .........T............FFFFFFF... \\
6 & ...........FTF.................. \\
7 & T......F.F.TFT.....F..FFFFFFTFF. \\
8 & F..................F..TTTTFTTTT. \\
9 & ........TT...................... \\
10 & FFF..TF.TT...................... \\
11 & FTT..FT......................... \\
12 & ...................F............ \\
13 & ..FF..F............F............ \\
14 & ..TT..T......................... \\
15 & ...T............................ \\
16 & ................................ \\
17 & ...T............................ \\
18 & ...T............................ \\
19 & ................................ \\
... & ................................ \\
47 & ................................ \\
\hline
\end{tabular}
}
\end{tiny}
\caption{Discovered fixed-value constraints in MD4 under Sasaki's Attack}
\label{table:4}
\end{table}


{\footnotesize \bibliographystyle{acm}

\bibliography{thesis}}


% \theendnotes

\end{document}
