% TEMPLATE for Usenix papers, specifically to meet requirements of
%  USENIX '05
% originally a template for producing IEEE-format articles using LaTeX.
%   written by Matthew Ward, CS Department, Worcester Polytechnic Institute.
% adapted by David Beazley for his excellent SWIG paper in Proceedings,
%   Tcl 96
% turned into a smartass generic template by De Clarke, with thanks to
%   both the above pioneers
% use at your own risk.  Complaints to /dev/null.
% make it two column with no page numbering, default is 10 point

% Munged by Fred Douglis <douglis@research.att.com> 10/97 to separate
% the .sty file from the LaTeX source template, so that people can
% more easily include the .sty file into an existing document.  Also
% changed to more closely follow the style guidelines as represented
% by the Word sample file.

% Note that since 2010, USENIX does not require endnotes. If you want
% foot of page notes, don't include the endnotes package in the
% usepackage command, below.

\documentclass[letterpaper,twocolumn,10pt]{article}
\usepackage{usenix,epsfig,endnotes,listings,amsmath,xcolor,courier,float}
\usepackage{hyperref,supertabular,amssymb}

\providecommand{\lxor}{\oplus}

\lstset{
  basicstyle=\ttfamily,
  columns=fullflexible,
  frame=single,
  breaklines=true,
  postbreak=\mbox{\textcolor{red}{$\hookrightarrow$}\space},
}

\begin{document}

%don't want date printed
\date{}

%make title bold and 14 pt font (Latex default is non-bold, 16 pt)
\title{\Large \bf The Neighborhood of a MD4 Collision }

\author{
{\rm Alexander Scheel}\\
\href{mailto:alexander.m.scheel@gmail.com}{alexander.m.scheel@gmail.com} \\
Iowa State University
\and
{\rm Eric Rozier}\\
\href{mailto:erozier@iastate.edu}{erozier@iastate.edu} \\
Iowa State University
}

\maketitle

% Use the following at camera-ready time to suppress page numbers.
% Comment it out when you first submit the paper for review.
% \thispagestyle{empty}


\subsection*{Abstract}

In 2005, Wang et al. published the details of a collision attack against the
MD4 hash function. We extend their work by showing that not
only is the attack state independent and differential independent, but also
that the neighborhood of the attack is populated with other classes of
collisions. We describe an efficient encoding of this problem into SAT,
and present example classes in the neighborhood of Wang's.

\section{Introduction}
The MD4 algorithm was developed by Ron Rivest around 1990 and is standardized
in RFC 1320 \cite{rfc1320}. While largely known to be insecure, and thus
having fallen out of use, its simple construction leads to fast implementations
and easy encoding as a SAT problem. An attack against a chosen-state MD4 was
published by Dobbertin in 1998 \cite{Dobbertin1998}. In 2004, Wang et al.
demonstrated a collision attack against the full MD4
\cite{cryptoeprint:2004:199}, before publishing the details in 2005
\cite{Wang2005}. With reduced time complexity compared to Dobbertin, this
class of collision went on to inspire other attacks such as Sasaki et al.,
which pushed the complexity of collisions to less than two MD4 evaluations
\cite{Sasaki2007}.

Outside of work published in 2006 by Mironov and Zhang \cite{Mironov2006},
the authors have found little evidence of the application of SAT solvers
to hash function collisions and MD4 specifically. The Mironov and Zhang work
was focused on the feasibility of such an encoding and the time for finding
new examples of collisions in the Wang et al. class.

We present the following four improvements in collisions.
\begin{enumerate}
    \itemsep0em
    \item Wang et al.'s attack is differential independent.
    \item Wang et al.'s attack is state independent.
    \item Wang et al.'s attack is highly malleable.
    \item The neighborhood of Wang et al.'s attack contains other collision classes.
\end{enumerate}

\section{Terminology}
We first begin by giving a brief set of terminology:

A \textbf{differential} is the signed XOR difference between two input blocks
to a hash function.

The \textbf{differential path} is the set of state transition constraints
which produce a full or partial collision.

The \textbf{constraints} of a collision are the necessary and/or sufficient
boolean formulas for generating a collision, typically as a function of the
state blocks.

A \textbf{state transition constraint} is a constraint on the intermediate
state variables, i.e., round variables. A constraint is \textbf{weak} if
it is a any difference, and a constraint is \textbf{strong} if it is a
signed difference. A equality constraint can either be weak or strong. This
is roughly equivalent to a differential path.

\textbf{Input block constraints}, \textbf{input state constraints}, and
\textbf{output state constraints} are all analogously defined as state
transition constraints.

A \textbf{property} of a collision is any constraint placed upon the
input block which exists independent from the constraints of a collision.

An attack is \textbf{constraint-based} if there exists a SAT encoding of the
attack which can generate multiple collisions.

A collision is \textbf{differential-independent} if there are multiple
input block differentials which satisfy the state transition constraints.

A collision is \textbf{malleable} if collisions can be found with different
properties such as ASCII or JSON input blocks, or with fixed internal text.

A collision is \textbf{state-independent} if there exist alternative
starting states which produce collisions. A state-independent collision
is termed \textbf{strong} if one block can collide under two or more starting
states.

A \textbf{collision class} is defined by the differential path and optionally
differential. The \textbf{neighborhood of a collision class} is a set of
alternative differential paths with low distance between differences.
For example, consider the differential path with $\Delta(I_{0 ... 48}) = 0$;
then the differential path with $\Delta(I_0) = 1$ has low distance to
$\Delta(I_0) = 0$, and hence is said to be in the neighborhood of
$\Delta(I_{0 ... 48})$.


\section{Notation}

As we are not concerned with the theoretical underpinnings of the SAT models,
nor will we use the model to provide general proofs, we use the following
notations. This notation is heavily influenced by our use of the $bc2cnf$
utility by Junttila \cite{circuits}.

A constraint, $c = (n, e)$ is a mapping between a name, $n$, and an equation,
$e$, of boolean clauses. These equations can be simple ($a \land b$) or complex
($a \lxor (c \land d) \lxor (a \lor b)$). A model, $S = (C, A)$, is an ordered
pair of a set, $C$, of constraints and a set, $A$, a set of names which must
evaluate to true. Note that input variables are implicit and thus left
unspecified.

We construct the base model for MD4 as follows. Let $s_{0} .. s_{127}$ be the
input state variables. Let $b_{0} .. b_{511}$ be the input block. We define
$i_0 .. i_{1535}$ as the bits of the intermediate state variables generated
via MD4, per RFC 1320 \cite{rfc1320}, and $o_{0} .. o_{128}$ as the resulting
output state. With the $s$'s and $b$'s left unspecified, the model for MD4
is thus $C_{MD4} = \{(i_0, e(i_0)), ..., (o_{128}, e(o_{128})) \}$. As we
are interested in collisions, we use $C_{MD4, h_1}$ to denote the construction
of MD4 with variables $i$, $o$, $s$, and $b$ prefixed with $h_1$.

We then define $I_{md4}$ to be the set of initial state values, with
$I_{md4, h_1}$ to be those state values with names prefixed by $h_1$. We also
define $C_{collision, h_1, h_2}$ to be the two-way collision between
$h_1$ and $h_2$; that is, all output bits are equal.

\section{Differential Independent}

We note that Wang et al.'s attack on MD4 is differential independent. Let
$C_{wangs, h_1, h_2}$ be the set of the state transition constraints on page 15
and sufficient constraints on page 16 \cite{Wang2005}. Let
$\bar{\Delta}_{wangs}$ be the negation of the specified differential on page 7
\cite{Wang2005}. Then we claim that
\begin{align*}
S = ( \{ C_{MD4, h_1} \cup C_{MD4, h_2} \cup C_{wangs, h_1, h_2} \cup
\bar{\Delta}_{wangs} \cup C_{collision, h_1, h_2} \cup I_{md4, h_1} \cup
I_{md4, h_2} \}, \{ C_{wangs}, \bar{\Delta}_{wangs},
C_{collision h_1, h_2} \})
\end{align*}
 is a model which encodes a collision between $h_1$
and $h_2$ satisfying the conditions of Wang's attack but with a diffferent
differential.

Claim: the model $S$ above is satisfiable. We give the following examples
as proof. 



By negating each found differential and re-running we can find all possible
differentials. This is a total of 238 differentials. We note the signed
block differences below. 



\section{State Independent}

We note that Wang et al.'s is strongly state independent. We construct a model
$S$ with four instances of MD4, $h_1$, $h_2$, $h_3$, $h_4$, which share an input
block and have a non-zero input state delta. 

We thus claim that
\begin{align*}
S = ( \{ C_{MD4, h_1}, C_{MD4, h_2}, C_{MD4, h_3}, C_{MD4, h_4},
C_{wangs, h_1, h_2}, C_{wangs, h_3, h_4}, \Delta_{wangs, h_1, h_2},
\Delta_{wangs, h_3, h_4}, C_{collision, h_1, h_2}, C_{collision, h_3, h_4},
\Delta_{any, h_1s, h_2s} \}, \{
C_{wangs, h_1, h_2}, C_{wangs, h_3, h_4}, \Delta_{wangs, h_1, h_2},
\Delta_{wangs, h_3, h_4}, C_{collision, h_1, h_2}, C_{collision, h_3, h_4},
\Delta_{any, h_1s, h_2s} \} )
\end{align*}
encodes the above properties and is satisifable. We reproduce several
examples below.




\section{Highly Malleable}
In the above style above, we note that Wang et al.'s is malleable, and highly
malleable under alternative differentials. We produce the following examples;
though we note that it is not possible with SAT prove that the attack is
fully malleable and can produce collisions with any property.


\section{Neighborhood}



{\footnotesize \bibliographystyle{acm}

\bibliography{thesis}}


% \theendnotes

\end{document}
