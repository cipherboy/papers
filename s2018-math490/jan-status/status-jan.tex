\documentclass[10pt,twocolumn,twoside]{pnas-new}
% Use the lineno option to display guide line numbers if required.
% Note that the use of elements such as single-column equations
% may affect the guide line number alignment.

\templatetype{pnasmathematics} % Choose template
% {pnasresearcharticle} = Template for a two-column research article
% {pnasmathematics} = Template for a one-column mathematics article
% {pnasinvited} = Template for a PNAS invited submission

\title{Status Update: End of January \\ MATH 490 - Independent Study}

% Use letters for affiliations, numbers to show equal authorship (if applicable) and to indicate the corresponding author
\author[a]{Alexander Scheel}

\affil[a]{Iowa State University}

% Please give the surname of the lead author for the running footer
\leadauthor{Scheel}

% Please add here a significance statement to explain the relevance of your work

% Please include corresponding author, author contribution and author declaration information

% Keywords are not mandatory, but authors are strongly encouraged to provide them. If provided, please include two to five keywords, separated by the pipe symbol, e.g:

\begin{abstract}
    End of January status update for MATH 490: Independent Study at Iowa State
    University. Proposes to study SHA-3 and outlines current and future areas
    of study and completed work.
\end{abstract}

\keywords{MATH 490 $|$ Status Report $|$ SHA-3 $|$ hash\_framework }

\begin{document}

% Optional adjustment to line up main text (after abstract) of first page with line numbers, when using both lineno and twocolumn options.
% You should only change this length when you've finalised the article contents.
\verticaladjustment{-2pt}

\maketitle
% \thispagestyle{firststyle}
% \ifthenelse{\boolean{shortarticle}}{\ifthenelse{\boolean{singlecolumn}}{\abscontentformatted}{\abscontent}}{}

% Bibliography
% \bibliography{pnas-sample}

\section{Introduction}

    The purpose of this independent study is to study cryptography from a
mathematical background. The target object is the Keccak hash function
as standardized by NIST as SHA-3 and its collision resistance. A cryptographic
hash function maps arbitrary length input strings to a fixed-length digest;
finding two elements of the input domain which map to the same output should
require on the order of $2^{\frac{l}{2}}$ amount of work, where $l$ is the
output length of the hash function. SHA-3 provides an excellent study due to
its relatively simple construction as compared to previous standardized hash
functions. The goal of this independent study is to better understand
cryptanalysis techniques such as differential cryptanalysis and computer-aided
proof techniques.

    The rest of this status report is organized as follows: section
\ref{sec:overview} details the construction of SHA-3, section
\ref{sec:literature} details some existing literature in the field, section
\ref{sec:completed} details the completed work, section \ref{sec:current}
details current tasks, and section \ref{sec:future} concludes the status
report with a list of future tasks and deliverables.

\section{Overview of SHA-3}\label{sec:overview}

    SHA-3 is the latest generation of hash function standardized by NIST in
FIPS-202, published in 2015 \cite{NIST202}. The team behind this function
includes Guido Bertoni, Joan Daemen, Micha{\"e}l Peeters, Gilles Van Assche,
and Ronny Van Keer \cite{KeccakTeam}, a European team. This hash function
differs significantly from the previously accepted construction of MD4, MD5,
SHA-1, and SHA-2: while the previous hash functions relied on the
Merkel-Damg{\aa}rd construction, SHA-3 uses a sponge-function design.

    The Merkel-Damg{\aa}rd construction for cryptographic hash functions
follows what has been termed the ARX construction: arithmetic-rotation-xor.
This construction transforms an initial state to an ouput state according to
a predefined series of round functions. Each round function is typically
different from the next, and each operates on a different portion of the block.
Assume a 32-bit hash function. Typically there are between 4 and 8 state
variables, in this case, 32-bit unsigned integers. The input block is split
into 32-bit chunks and expanded according to the defined block schedule.
This block schedule can either be simple (in the case of MD4 and MD5, simply
permutations of the ordering of the chunks) or complex (in the case of SHA-1
and SHA-2, which involve xor-ing various chunks together to produce additional
chunks).

    Then, starting from the initial state, each round applies the corresponding
round function to the current state and the current block chunk, producing a
new state variable. The round function is made up of several additions,
at least one rotation and an xor; hence the term ARX construction.
This state variable is then used to replace an element of
the current state (directly in the case of MD4 and MD5, else in some fixed
pattern in SHA-1 and SHA-2). Lastly, after the last block chunk has been
processed, the initial state and the final state are added together to produce
the output state. Multiple blocks are then handled by assuming this output
state as input to the next block, and some suitable padding scheme is applied
to the input before splitting into blocks. In the case of MD4, MD5, SHA-1,
SHA-2 256 and SHA-2 512, all of the state variables are directly outputted.
Therefore, an extension attack is possible: given any input $B$, which
produced output $O$, it holds that all $C$ produce $h(O, C) = h(IV, B || C)$,
where $IV$ is the default initial state value.

    In contrast, Keccak uses a construction they call a sponge construction.
Rather than several independent state variables, Keccak uses a single
array as state, of size $25 \times 2^{l}$ for $0 \leq l \leq 6$. This is
conceptualized as a $5 \times 5 \times 2^{l}$ cuboid. Blocks are mixed in
once, via a simple xor, and the initial state is the all zero array. Then,
a round function is permuted 24 times. This round function is composed of
five functions: $\theta, \rho, \pi, \chi, \iota$. Each function is a bijection
on the set of all state arrays. Thus, after 24 rounds are applied, a portion
of the resulting state is specified as output. If multiple blocks are given,
the entire state is used as input, with the next block xored in, before
applying the round function.

    SHA-3 standardized $l=6$, producing a state array of 1600 bits. Further,
since SHA-2 standardized several different security margins, (224, 256,  384,
and 512), SHA-3 did as well. Take the security margin to be $b$ bits.
Then the input block size is $1600 - 2\times b$ and $b$ bits of the resulting
state is taken as output. While this prevents length extension attacks on
every block, this opens the possibility for a specialized extension attack
with lower possibility. Suppose $S_0$ is the initial state after mixing
in a block. If $\text{keccak-}p(S_0)$ has the same lower $2 \times b$ bits
as $S_0$, then $h(S_0 \oplus C) = h(S_0 || C)$, for all messages $C$.



\section{Review of Literature}\label{sec:literature}

    The canonical reference for SHA-3 is FIPS-202 \cite{NIST202}. However,
there are several useful supplementary materials to use in conjunction with
FIPS-202. The first is the core Keccak Reference: published by the Keccak team,
this contains extended information underlying their choice of functions,
properties of these functions, documentation on inverse functions, and external
references \cite{Keccak3}. Supplementing this is the Keccak Code Package,
which contains many implementations of Keccak and test cases for several bit
widths \cite{KeccakCodePackage}. Lastly, the PhD thesis of J. Daemen seems
to have heavily influenced the construction of Keccak, and provides extensive
reference material \cite{DaemenThesis}.

    Of the attacks I've looked at, the work of E. Homsirikamol, et. al,
is most relevant to my existing work \cite{cryptoeprint:2012:421}. They
directly applied SAT solvers to find collisions in Keccak. However, they used
only the most simple model and were only able to find a two-round collision
within 30 hours. Thus it serves more as a benchmark than as useful
cryptanalysis. The work of P. Morawiecki and M. Srebrny also looks at applying
SAT solvers to Keccak, but explicitly focused on preimage attacks
\cite{cryptoeprint:2010:285}. In this instance, it appears the authors only
considered a direct preimage attack with little other work towards splitting
the work between multiple servers or doing advanced exploration of possible
paths. However, they did appear to build a toolkit, CryptLogVer, for studying
this problem.

\section{Completed Work} \label{sec:completed}

    Completed is an initial survey of the literature on the subject of the
construction of SHA-3 and attacks against it. Further, an implementation
of SHA-3 has been added to my hash\_framework package \cite{hf}. This has been
tested against the vectors in the Keccak Code Project \cite{KeccakCodePackage},
and is shown correct.

    Several initial attack vectors have been considered as well. Towards
studying the specialized extension attack, an initial pass trying to bound the
error in the remaining bits. However, up to an input difference of 5, no
alternate paths have been found producing a net-zero difference in the tail.
This has yet to be studied on reduced width versions of Keccak.

Towards studying collision attacks, a similar error-bounding property has been
studied on front prefixes. Given that there are $k$ bits of difference in the
input, and given an input margin of $m$ (where the remaining 1600 - m bits
are all zero), what is the largest $M$ such that, after applying
only the first round, the difference between the two blocks is zero in the
first $M$ bits. This has been exhaustively studied on the $l=3$ and $l=4$
bit-width versions of Keccak, and partially on the full width ($l=6$). The
results are startling: for small, even $k$, $M$ quickly grows large, whereas
for odd, small $k$, $M$ remains fixed at a relatively small number. Thus
there is the potential to exploit this. However, as $k$ grows large, the
odd and even values of $k$ converge in structure. This can seen in the links
attached to this email.

\section{Current Work} \label{sec:current}

    Currently, I am rewriting my job distribution mechanism to work more
efficiently on larger deployments such as an HPC cluster or AWS cloud
instances. This will add several desirable features: better fault tolerance,
handling job distribution automatically and prioritizing between several
tasks, and devoting a set number of cores per task. Each task will then have
many jobs and statistics about them can be collected and centralized (such as
number of variables, time to solve, etc.). This would provide the basis for
more closely analyzing the performance of the entire system.


\section{Future Work} \label{sec:future}

    After this rewrite, I'll look at exploring a few areas: analyzing the
impact of each of the sub-round functions ($\theta, \rho, \pi, \chi, \iota$)
on various differential paths, continuing the above work on analyzing
input and output margins on $k$-bit differences, and looking at total, given
$k$ bits of input difference, what possible combinations of output differences
are possible.

    Further, I'll implement HPC-specific workloads. The most obvious starting
point is to analyzing small versions of Keccak ($l=4$) for two-variable
correlations across each of the several rounds. Further, studying the
per-round correlation of initial input differential on resulting intermediate
round differentials would be an interesting result as well, for small $l$.

    Once these properties have been more closely explored, there is potential
for proving theorems based on the results. However, I think it is important to
get some initial results from these various areas to see how they all fit
together and what avenues are most likely to help attack Keccak.


\section{Bibliography} \label{sec:bibliography}

\bibliography{thesis}

\end{document}
