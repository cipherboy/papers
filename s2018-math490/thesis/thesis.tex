\documentclass[10pt,twocolumn,twoside]{pnas-new}
% Use the lineno option to display guide line numbers if required.
% Note that the use of elements such as single-column equations
% may affect the guide line number alignment.

\templatetype{pnasmathematics} % Choose template
% {pnasresearcharticle} = Template for a two-column research article
% {pnasmathematics} = Template for a one-column mathematics article
% {pnasinvited} = Template for a PNAS invited submission

\title{Keccak Thesis}

% Use letters for affiliations, numbers to show equal authorship (if applicable) and to indicate the corresponding author
\author{Alexander Scheel}

\affil{Iowa State University}

% Please give the surname of the lead author for the running footer
\leadauthor{Scheel}

% Please add here a significance statement to explain the relevance of your work

% Please include corresponding author, author contribution and author declaration information
\correspondingauthor{E-mail: alexander.m.scheel\@gmail.com}

% Keywords are not mandatory, but authors are strongly encouraged to provide them. If provided, please include two to five keywords, separated by the pipe symbol, e.g:

\begin{abstract}
    % TODO
    TODO
\end{abstract}

\keywords{MATH 490 $|$ Advisor: Dr. Clifford Bergman $|$ SHA-3 $|$ hash\_framework }

\begin{document}

% Optional adjustment to line up main text (after abstract) of first page with line numbers, when using both lineno and twocolumn options.
% You should only change this length when you've finalised the article contents.
\verticaladjustment{-2pt}

\maketitle
% \thispagestyle{firststyle}
% \ifthenelse{\boolean{shortarticle}}{\ifthenelse{\boolean{singlecolumn}}{\abscontentformatted}{\abscontent}}{}

% Bibliography
% \bibliography{pnas-sample}

\section{A Series of Introductions} \label{sec:intro}
    This paper presents the work of the author towards the completion of the
requirements for the MATH 490 Independent Study course under Dr. Clifford
Bergman at Iowa State University of Science and Technology. This work was an
extension of the author's Honors Project under Dr. Eric Bergman and utilized
the resulting
\href{https://github.com/cipherboy/hash_framework}{hash\_framework} project.
Additional artifacts related to this project can be seen in the
\href{https://github.com/cipherboy/keccak-attacks}{keccak-attacks} repository.

    Within this section are a series of introductions which provide necessary
background on the topics of cryptography, hash functions, the development of
Keccak/SHA-3, and its structure. While certain sections can be skipped if the
reader has the prerequisite knowledge, hopefully all readers find the material
engaging and useful. Following these introductions, this paper presents the
analysis of the Keccak/SHA-3 hash function before concluding with a final
evaluation and further work.

% TODO
% More information about structure of analysis section.


\subsection{Introduction on the Topics of Cryptography} \label{sec:i:crypto}

    While there have been many applications of mathematics, few are as
demanding and shrouded in secrecy as that of cryptography. Cryptography exists
because of the fundamental need of civilizations, governments, and individuals
to keep secrets secure from devoted adversaries.

% TODO Outline
%    - Historic
%        - Ciphers vs Codes
%
%    - Modern
%        - Symmetric
%        - Asymmetric
%        - Hash Functions
%        - Building Protocols

\subsection{Introduction on the Topics of Hash Functions} \label{sec:i:hash}

    Within cryptography's collection of algorithms, few are are as useful as
hash functions have proven to be to cryptographers and non-cryptographers
alike. A hash function maps arbitrary length binary strings to binary strings
of a fixed length.

% TODO Outline
%   - Properties
%   - History of Development
%   - History of Attacks
%   - Modern Theory and Structure

\subsection{Introduction on the Development of Keccak/SHA-3} \label{sec:i:keccak}

    FIPS 202 \cite{NIST202}.

% TODO Outline
%    - Authors Background (keccak team)
%    - Function design goals
%    - NIST Contest
%    - NIST standardization controversy? (sources?)

\subsection{Introduction on the Structure of SHA-3} \label{sec:i:structure}

    SHA-3 consists of two parts: a core permutation function, KECCAK-$f$, and
a domain extender, the KECCAK sponge function.

% TODO Outline
%   - Sponge Function
%   - State Array
%   - Five functions
%   - Keccak-$f$

\section{Mathematical Properties of the Five Round Functions} \label{sec:properties}

% TODO Outline
%   - Outline following subsections

\subsection{Bijectivity} \label{sec:p:bijectivity}

% TODO Outline
%   - Reason why bijectivity is important
%   - Impact on attacks?

\subsubsection{Of $\theta$} \label{sec:p:b:t}

% TODO Outline
%   - Theta -> XOR with 11 different values; linear equations?
%   - What sizes are bijective?
%   - Discussion in literature

\subsubsection{Of $\rho$} \label{sec:p:b:r}

% TODO Outline
%   - Obvious permutation of bit orders

\subsubsection{Of $\pi$} \label{sec:p:b:p}

% TODO Outline
%   - Obvious: permutation of bit orders

\subsubsection{Of $\chi$} \label{sec:p:b:c}

% TODO Outline
%   - Harder, discussion in literature.

\subsubsection{Of $\iota$} \label{sec:p:b:i}

% TODO Outline
%   - Obvious: XOR with fixed value.



\subsection{Order of the Permutation} \label{sec:p:order}

% TODO Outline
%   - Outline why interesting
%   - Attack on PRF/XOF -> low cycle size
%   - Values over 2^(25w) indicate PRF has smaller upper bound for actual
%       entropy
%   - T R P C I

\subsection{Evaluation of the Orders of Composition of Permutations}

% TODO Outline
%   - w=1 evaluation
%   - Evaluation of different orderings of trpci

\subsection{Inverse of the Permutation} \label{sec:p:inverse}

% TODO Outline
%   - Outline simplification of f^{-1}
%   - Closed form for theta, chi
%   - Give algorithm for any w?

\subsection{XOR ($\oplus$) Distributivity} \label{sec:p:xor}

% TODO Outline
%   - Theta, Rho, Pi - free
%   - Chi, Iota - no
%   - Makes analysis harder, which is good

\subsection{Generalizations of $\theta$ and $\chi$} \label{sec:p:generalizations}

% TODO Outline
%   - Evaluation of effect?
%   - Pending eval?


\subsection{Choice of Parameters and Ordering of Composition} \label{sec:p:generalizations}
% TODO Outline
%   - Pending evaluation?


\section{Marginal and Differential Properties of the Five Round Functions} \label{sec:differentials}

% TODO Outline
%   - Overview of why it matters

\subsection{Margin Properties} \label{sec:d:margin}

% TODO Outline
%   - Overview of why it matters - one round
%   - Can see effect of rho in w increasing
%   - Graphics

\subsection{Differential Properties} \label{sec:d:diff}

% TODO Outline
%   - Graphics
%   - Impact of importance
%   - Show composition order impact if any?

\subsection{Input Margin Impact on Differential Properties} \label{sec:d:input}

% TODO Outline
%   - Graphics
%   - Impact and importance

\subsection{Output Margin Impact on Differential Properties} \label{sec:d:output}

% TODO Outline
%   - Graphics
%   - Impact and importance


\section{Exhaustive Collision Searches} \label{sec:search}

% TODO Outline
%   - Importance
%   - Finding patterns hard
%   - No obvious connection for how to extend w=1 -> w=2
%   - Searches limited

\section{Fixed Point Attacks} \label{sec:fixed}

% TODO Outline
%   - Outline subsections

\subsection{Full Fixed Points} \label{sec:f:full}

% TODO Outline
%   - Probable due to order analysis
%   - Impact, attack

\subsection{Partial Fixed Points} \label{sec:f:partial}

% TODO Outline
%   - More probable
%   - Impact, attack


\section{Conclusions and Further Work} \label{sec:conclusion}

% TODO Outline
%   - SHA-3 Secure, but introduces new attack vectors
%   - Need more in the way of LC for SHA-3-esq functions
%_  - Needs permutation-based cryptanalysis
%   - Differential cryptanalysis not likely to be of much use

\section{Bibliography} \label{sec:bibliography}

\bibliography{thesis}

\end{document}
