\documentclass[10pt,twocolumn,twoside]{pnas-new}
% Use the lineno option to display guide line numbers if required.
% Note that the use of elements such as single-column equations
% may affect the guide line number alignment.

\templatetype{pnasmathematics} % Choose template
% {pnasresearcharticle} = Template for a two-column research article
% {pnasmathematics} = Template for a one-column mathematics article
% {pnasinvited} = Template for a PNAS invited submission

\title{Keccak Thesis}

% Use letters for affiliations, numbers to show equal authorship (if applicable) and to indicate the corresponding author
\author{Alexander Scheel}

\affil{Iowa State University}

% Please give the surname of the lead author for the running footer
\leadauthor{Scheel}

% Please add here a significance statement to explain the relevance of your work

% Please include corresponding author, author contribution and author declaration information
\correspondingauthor{E-mail: alexander.m.scheel\@gmail.com}

% Keywords are not mandatory, but authors are strongly encouraged to provide them. If provided, please include two to five keywords, separated by the pipe symbol, e.g:

\begin{abstract}
    TODO
\end{abstract}

\keywords{MATH 490 $|$ Advisor: Dr. Clifford Bergman $|$ SHA-3 $|$ hash\_framework }

\begin{document}

% Optional adjustment to line up main text (after abstract) of first page with line numbers, when using both lineno and twocolumn options.
% You should only change this length when you've finalised the article contents.
\verticaladjustment{-2pt}

\maketitle
% \thispagestyle{firststyle}
% \ifthenelse{\boolean{shortarticle}}{\ifthenelse{\boolean{singlecolumn}}{\abscontentformatted}{\abscontent}}{}

% Bibliography
% \bibliography{pnas-sample}

\section{A Series of Introductions} \label{sec:intro}
    This paper presents the work of the author towards the completion of the
requirements for the MATH 490 Independent Study course under Dr. Clifford
Bergman at Iowa State University of Science and Technology. This work was an
extension of the author's Honors Project under Dr. Eric Bergman and utilized
the resulting
\href{https://github.com/cipherboy/hash_framework}{hash\_framework} project.
Additional artifacts related to this project can be seen in the
\href{https://github.com/cipherboy/keccak-attacks}{keccak-attacks} repository.

    Within this section are a series of introductions which provide necessary
background on the topics of cryptography, hash functions, the development of
Keccak/SHA-3, and its structure. While certain sections can be skipped if the
reader has the prerequisite knowledge, hopefully all readers find the material
engaging and useful. Following these introductions, this paper presents the
analysis of the Keccak/SHA-3 hash function before concluding with a final
evaluation and further work.


\subsection{Introduction on the Topics of Cryptography} \label{sec:i:crypto}

    While there have been many applications of mathematics, few are as
demanding and shrouded in secrecy as that of cryptography. Cryptography exists
because of the fundamental need of civilizations, governments, and individuals
to keep secrets secure from devoted adversaries.

\subsection{Introduction on the Topics of Hash Functions} \label{sec:i:hash}

    Within cryptography's collection of algorithms, few are are as useful as
hash functions have proven to be to cryptographers and non-cryptographers
alike. A hash function maps arbitrary length binary strings to binary strings
of a fixed length.

\subsection{Introduction on the Development of Keccak/SHA-3} \label{sec:i:keccak}

    FIPS 202 \cite{NIST202}.

\subsection{Introduction on the Structure of SHA-3} \label{sec:i:structure}





\section{Mathematical Properties of the Five Round Functions} \label{sec:properties}

\subsubsection{Bijectivity} \label{sec:p:bijectivity}

\subsubsection{Order of the Permutation} \label{sec:p:order}

\subsubsection{Inverse of the Permutation} \label{sec:p:inverse}

\subsubsection{XOR ($\oplus$) Distributivity} \label{sec:p:xor}

\subsubsection{Generalizations of $\theta$ and $\chi$} \label{sec:p:generalizations}

\subsubsection{Choice of Parameters for $\theta$ and $\chi$} \label{sec:p:generalizations}




\section{Marginal and Differential Properties of the Five Round Functions} \label{sec:differentials}

\subsection{Margin Properties} \label{sec:d:margin}

\subsection{Differential Properties} \label{sec:d:diff}

\subsection{Input Margin Impact on Differential Properties} \label{sec:d:input}

\subsection{Output Margin Impact on Differential Properties} \label{sec:d:output}




\section{Exhaustive Collision Searches} \label{sec:search}

\section{Fixed Point Attacks} \label{sec:fixed}

\subsection{Full Fixed Points} \label{sec:f:full}

\subsection{Partial Fixed Points} \label{sec:f:partial}




\section{Conclusions and Further Work} \label{sec:conclusion}

\section{Bibliography} \label{sec:bibliography}

\bibliography{thesis}

\end{document}
